% Options for packages loaded elsewhere
\PassOptionsToPackage{unicode}{hyperref}
\PassOptionsToPackage{hyphens}{url}
\PassOptionsToPackage{dvipsnames,svgnames,x11names}{xcolor}
%
\documentclass[
  letterpaper,
  DIV=11,
  numbers=noendperiod]{scrreprt}

\usepackage{amsmath,amssymb}
\usepackage{iftex}
\ifPDFTeX
  \usepackage[T1]{fontenc}
  \usepackage[utf8]{inputenc}
  \usepackage{textcomp} % provide euro and other symbols
\else % if luatex or xetex
  \usepackage{unicode-math}
  \defaultfontfeatures{Scale=MatchLowercase}
  \defaultfontfeatures[\rmfamily]{Ligatures=TeX,Scale=1}
\fi
\usepackage{lmodern}
\ifPDFTeX\else  
    % xetex/luatex font selection
\fi
% Use upquote if available, for straight quotes in verbatim environments
\IfFileExists{upquote.sty}{\usepackage{upquote}}{}
\IfFileExists{microtype.sty}{% use microtype if available
  \usepackage[]{microtype}
  \UseMicrotypeSet[protrusion]{basicmath} % disable protrusion for tt fonts
}{}
\makeatletter
\@ifundefined{KOMAClassName}{% if non-KOMA class
  \IfFileExists{parskip.sty}{%
    \usepackage{parskip}
  }{% else
    \setlength{\parindent}{0pt}
    \setlength{\parskip}{6pt plus 2pt minus 1pt}}
}{% if KOMA class
  \KOMAoptions{parskip=half}}
\makeatother
\usepackage{xcolor}
\setlength{\emergencystretch}{3em} % prevent overfull lines
\setcounter{secnumdepth}{5}
% Make \paragraph and \subparagraph free-standing
\makeatletter
\ifx\paragraph\undefined\else
  \let\oldparagraph\paragraph
  \renewcommand{\paragraph}{
    \@ifstar
      \xxxParagraphStar
      \xxxParagraphNoStar
  }
  \newcommand{\xxxParagraphStar}[1]{\oldparagraph*{#1}\mbox{}}
  \newcommand{\xxxParagraphNoStar}[1]{\oldparagraph{#1}\mbox{}}
\fi
\ifx\subparagraph\undefined\else
  \let\oldsubparagraph\subparagraph
  \renewcommand{\subparagraph}{
    \@ifstar
      \xxxSubParagraphStar
      \xxxSubParagraphNoStar
  }
  \newcommand{\xxxSubParagraphStar}[1]{\oldsubparagraph*{#1}\mbox{}}
  \newcommand{\xxxSubParagraphNoStar}[1]{\oldsubparagraph{#1}\mbox{}}
\fi
\makeatother


\providecommand{\tightlist}{%
  \setlength{\itemsep}{0pt}\setlength{\parskip}{0pt}}\usepackage{longtable,booktabs,array}
\usepackage{calc} % for calculating minipage widths
% Correct order of tables after \paragraph or \subparagraph
\usepackage{etoolbox}
\makeatletter
\patchcmd\longtable{\par}{\if@noskipsec\mbox{}\fi\par}{}{}
\makeatother
% Allow footnotes in longtable head/foot
\IfFileExists{footnotehyper.sty}{\usepackage{footnotehyper}}{\usepackage{footnote}}
\makesavenoteenv{longtable}
\usepackage{graphicx}
\makeatletter
\newsavebox\pandoc@box
\newcommand*\pandocbounded[1]{% scales image to fit in text height/width
  \sbox\pandoc@box{#1}%
  \Gscale@div\@tempa{\textheight}{\dimexpr\ht\pandoc@box+\dp\pandoc@box\relax}%
  \Gscale@div\@tempb{\linewidth}{\wd\pandoc@box}%
  \ifdim\@tempb\p@<\@tempa\p@\let\@tempa\@tempb\fi% select the smaller of both
  \ifdim\@tempa\p@<\p@\scalebox{\@tempa}{\usebox\pandoc@box}%
  \else\usebox{\pandoc@box}%
  \fi%
}
% Set default figure placement to htbp
\def\fps@figure{htbp}
\makeatother

\KOMAoption{captions}{tableheading}
\makeatletter
\@ifpackageloaded{bookmark}{}{\usepackage{bookmark}}
\makeatother
\makeatletter
\@ifpackageloaded{caption}{}{\usepackage{caption}}
\AtBeginDocument{%
\ifdefined\contentsname
  \renewcommand*\contentsname{Table of contents}
\else
  \newcommand\contentsname{Table of contents}
\fi
\ifdefined\listfigurename
  \renewcommand*\listfigurename{List of Figures}
\else
  \newcommand\listfigurename{List of Figures}
\fi
\ifdefined\listtablename
  \renewcommand*\listtablename{List of Tables}
\else
  \newcommand\listtablename{List of Tables}
\fi
\ifdefined\figurename
  \renewcommand*\figurename{Figure}
\else
  \newcommand\figurename{Figure}
\fi
\ifdefined\tablename
  \renewcommand*\tablename{Table}
\else
  \newcommand\tablename{Table}
\fi
}
\@ifpackageloaded{float}{}{\usepackage{float}}
\floatstyle{ruled}
\@ifundefined{c@chapter}{\newfloat{codelisting}{h}{lop}}{\newfloat{codelisting}{h}{lop}[chapter]}
\floatname{codelisting}{Listing}
\newcommand*\listoflistings{\listof{codelisting}{List of Listings}}
\makeatother
\makeatletter
\makeatother
\makeatletter
\@ifpackageloaded{caption}{}{\usepackage{caption}}
\@ifpackageloaded{subcaption}{}{\usepackage{subcaption}}
\makeatother

\usepackage{bookmark}

\IfFileExists{xurl.sty}{\usepackage{xurl}}{} % add URL line breaks if available
\urlstyle{same} % disable monospaced font for URLs
\hypersetup{
  colorlinks=true,
  linkcolor={blue},
  filecolor={Maroon},
  citecolor={Blue},
  urlcolor={Blue},
  pdfcreator={LaTeX via pandoc}}


\author{}
\date{}

\begin{document}

\renewcommand*\contentsname{Table of contents}
{
\hypersetup{linkcolor=}
\setcounter{tocdepth}{2}
\tableofcontents
}

\bookmarksetup{startatroot}

\chapter{Explore the Book}\label{explore-the-book}

\href{SiteHistory.qmd}{\pandocbounded{\includegraphics[keepaspectratio]{images/thumbs/site.jpg}}}

\subsection{\texorpdfstring{\href{SiteHistory.qmd}{Site
Description}}{Site Description}}\label{site-description}

An overview of the landscape, watersheds, and forest structure of
Hubbard Brook.

\bookmarksetup{startatroot}

\chapter{Site, History, and Research
Approaches}\label{site-history-and-research-approaches}

Chapter Editor(s): Timothy Fahey

For over 60 years scientists have been studying the dynamics of forests
and linked aquatic ecosystems in the Hubbard Brook Experimental Forest,
New Hampshire, USA. The Hubbard Brook Ecosystem Study (HBES) is an
ongoing effort to understand the ecology, hydrology, energetics and
biogeochemistry of this temperate forest ecosystem. The synthesis that
follows provides an overview of several components of the HBES
interpreted in light of current understanding at the level of the
advanced student. The objective is to provide interested researchers
with a succinct summary that will be helpful for guiding research and
education efforts.

\begin{figure}

\centering{

\pandocbounded{\includegraphics[keepaspectratio]{figs/Intro/HubbardBrookMap-905x640.png}}

}

\caption{\label{fig-map}Hubbard Brook Experimental Forest Map}

\end{figure}%

\section{Site Description}\label{site-description-1}

The Hubbard Brook Experimental Forest (HBEF) is located in the southern
part of the White Mountain National Forest (WMNF) in central New
Hampshire (Figure~\ref{fig-map}). The HBEF is characteristic of much of
the White Mountain National Forest (Fahey et al.~2015) The National
Forest has hilly, occasionally steep topography; coarse, acidic,
glacially-derived soils; bedrock dominated by metamorphic rock of
igneous and sedimentary origin; northern hardwood forests on lower
slopes and spruce-fir forests at higher elevations (above ca. 800 m).

\section{Climate}\label{climate}

The continental climate at the HBEF features long, cold winters and mild
to cool summers (see Climate chapter). Major air flow over the forest is
either (1) continental polar air from subarctic North America (the
predominant direction), (2) maritime tropical air from the Caribbean and
Gulf of Mexico from the south or southwest, or (3) maritime air from the
North Atlantic out of the east or northeast. In spite of the proximity
of Hubbard Brook to the ocean (116 km), the climate is predominantly
continental.

Annual precipitation averages about 1,400 mm, of which about one third
to one quarter is snow. A snowpack usually persists from mid-December
until mid-April, with a peak depth in March of about 1,000 mm, having
about 250 to 300 mm of water content.

Winters are long and cold. January averages about -9oC, and long periods
of low temperatures from -12˚C to -18˚C are common. Even though
temperatures are low most of the time, occasional midwinter thaws result
in elevated streamflow. Short, cool summers are the rule. The average
July temperature is 18oC.

The average number of days without killing frost is 145; however, the
growing season for trees is considered to be from 15 May, the
approximate time of full leaf development, to 15 September, when the
leaves begin to fall.

The estimated annual evapotranspiration (ET) is about 500 mm, determined
by difference between precipitation and streamflow. This calculation is
a reasonable approach for Hubbard Brook because of the apparent minimal
deep seepage and annual removal of summer soil-water deficits by
autumnal rains and spring snowmelt.

\section{Geology}\label{geology}

The eastern portion of the Experimental Forest (watersheds 1-6, and 9
included; Figure 1) is underlain by a complex assemblage of
metasedimentary and igneous rocks. The major map unit is the Silurian
Rangeley Formation, consisting of quartz mica schist and quartzite
interbedded with sulfidic schist and calc-silicate granulite. Originally
deposited as mudstones, sandstones and conglomerates, these rocks have
been metamorphosed to sillimanite grade and have undergone four stages
of deformation. Deformation style evident in outcrops is primarily tight
isoclinal folds. However joints, slickensides and mylonites indicate
brittle deformation as well. The metamorphic rocks were later intruded
by a variety of igneous rocks including the Devonian Concord Granite,
pegmatites, and Mesozoic diabase and lamprophyre dikes. The western
portion of the forest (portions of watersheds 7 and 8 included) is
underlain by the Devonian Kinsman Granodiorite, a foliated granitic rock
with megacrysts of potassium feldspar.

Continental glaciers, which blanketed the region during the Pleistocene
and retreated some 13,000 years ago, removed most preexisting soils.
Glacial movement was primarily in a southeasterly direction as indicated
by striations on bedrock surfaces, and by fragments of rocks in the till
which are typical of bedrock to the northwest of the Hubbard Brook
Valley. Materials deposited by the glacier are highly variable in degree
of sorting and grain size, ranging from clays to 10m diameter boulders.
The depth of glacial deposits ranges from zero on ridgetops and in
stream valleys (resulting in bedrock outcrops) to 50m in the vicinity of
Mirror Lake. Poorly sorted glacial till, commonly 2m thick, covers the
bedrock in most of the valley. Ice contact terraces in the lower valley,
consisting of well sorted sands and gravels, are typically tens of
meters thick.

\section{Soils}\label{soils}

In the HBEF soils are predominantly well-drained Spodosols, more
specifically, Typic Haplorthods derived from glacial till, with sandy
loam textures. There are no residual soils, (i.e., derived from
weathered bedrock). Principal soil series are the sandy loams of the
Berkshire series, along with the Skerry, Becket, and Lyman series. These
soils are acidic (pH about 4.5 or less) and relatively infertile
(Table~\ref{tbl-soil-properties}). A 20- to 200-mm thick forest floor
layer is usually present. Soils on the ridgetops may consist of a thin
accumulation of organic matter, resting directly on bedrock.

\begin{longtable}[]{@{}
  >{\raggedright\arraybackslash}p{(\linewidth - 4\tabcolsep) * \real{0.3333}}
  >{\raggedright\arraybackslash}p{(\linewidth - 4\tabcolsep) * \real{0.3333}}
  >{\raggedright\arraybackslash}p{(\linewidth - 4\tabcolsep) * \real{0.3333}}@{}}
\caption{Summary soils data for Watershed 5 at the HBEF (after Johnson
et al.~1991a,b).}\label{tbl-soil-properties}\tabularnewline
\toprule\noalign{}
\endfirsthead
\endhead
\bottomrule\noalign{}
\endlastfoot
\textbf{Soil Type} & Typic Haplorthod; Typic Dystrochrept & \\
\textbf{Soil Series} & Berkshire; Skerry; Becket; Lyman; Tunbridge & \\
\textbf{Property} & \textbf{Surface / Organic Layer} & \textbf{Mineral
Soil} \\
---- & ---- & ---- \\
Avg. Depth (cm) & 6.9 & 50.3 \\
Mass (kg/m²) & 8.8 & 317.0 \\
Soil Organic Matter (\%) & 60.0 & 10.0 \\
pH in Water & 3.9 (Oa Horizon) & 4.3 \\
Cation Exchange Capacity (cmol/kg) & 18.0 (Oa Horizon) & 5.0 \\
Base Saturation (\%) & 50.0 (Oa Horizon) & 12.0 \\
\end{longtable}

The separation between the pedogenic zone and the virtually unweathered
till and bedrock below is distinct. Depth to the C horizon averages
about 0.6 m. At various places in the Forest, the C horizon exists as an
impermeable pan. These layers restrict root development and water
movement. Rocks of all sizes are scattered throughout the soil profile.
In many locations boulder fields are prominent features.

A prominent feature of the surface topography throughout the HBEF is the
rough pit-and-mound appearance caused by the uprooting of trees.
Uprooting mixes mineral soil from lower mineral horizons with
nutrient-rich organic surface layers or it can deposit the lower mineral
layers directly on top of the forest floor humus layers without mixing,
creating buried horizons. This natural disturbance changes seedbed
conditions for regenerating species, and affects weathering and
biogeochemical cycles.

\section{Streams}\label{streams}

Although stream channels occupy only 1\% of the land area, the streams
play an important role in many processes throughout the HBEF. The
numerous streams in the HBEF range from small ephemeral channels that
often dry up during summer to a large perennial 5th-order stream
(Hubbard Brook). Because of the shallow soil depths and high soil
porosity, the stream channels quickly swell during large storms (e.g.,
80 mm or more). As much as 60 to 80\% of storm precipitation can pass
through the stream as storm flow.

Streamwater is usually characterized by low concentrations of suspended
materials due to the coarse-grained texture of the soil. Mineral and
organic particulate material is suspended and carried in storm periods,
but once the flow recedes these materials quickly settle, leaving the
stream with low concentrations of suspended solids. About 28 kg/ha of
particulate matter (both inorganic and organic) are transported out of
the watershed by streams each year.

Most of the stream channels have exposed bedrock at some locations.
However, the first to third order streams have channels primarily made
up of mineral and organic particulate matter lodged behind organic
debris dams. These dams form a stair-step pattern in stream channels,
and they play important roles in regulating many physical, chemical, and
biological processes in the streams.

\section{Historical Perspective}\label{historical-perspective}

The Hubbard Brook Experimental Forest was established by the U.S.
Department of Agriculture (USDA) Forest Service in 1955 as a major
center for hydrologic research in New England. The initial size
delineated for research was a 3,037-ha, bowl-shaped Valley (Figure 1).

During the first eight years following the establishment of the HBEF,
the Northeastern Research Station of the USDA Forest Service developed a
network of precipitation and stream-gauging stations, and installed
weather instrumentation, as well as soil and vegetation monitoring sites
on small experimental watersheds. Data from these installations combined
with several initial studies formed the hydrometeorologic foundation for
much of the future research at the HBEF. The major emphasis in these
early studies was to determine the impact of forest land management on
water yield,water quality, and flood flow.

The Hubbard Brook Ecosystem Study originated in 1960 with the idea of
using a small watershed approach to study element flux and cycling. The
initial development of the HBES was slow and deliberate. The entire
effort during the first two years, 1963-1965, was conducted by three
scientists and one technician at Dartmouth College and three scientists
and one technician from the USDA Forest Service (USDA-FS). At that time,
there were no precedents to follow since comprehensive studies of
ecosystems had not been initiated. Using the small watershed approach,
studies of element-hydrologic interactions were conducted to form a
basis for subsequent process-level and experimental research. In this
regard, the investigators were fortunate to rather quickly develop
quantitative element budgets for replicated ecosystems. These results
provided insight into the function of natural ecosystems and helped
focus future lines of research. It was agreed that slow growth would be
more manageable and would allow for interaction among all senior
investigators, ensuring proper coordination and development of the
overall study. Research problems that were timely and particularly
pertinent to the overall research goals of the study were identified.
From the beginning, the HBES has emphasized the value of knowledge
derived from cooperative research. At the same time, individual research
freedom has always been encouraged among both cooperating scientists and
graduate students. This policy has been largely responsible for the
intellectual growth of the HBES, as well as its role as a center for
undergraduate and graduate education in ecosystem science.

HBES has developed into a relatively complex matrix of projects
involving a large number of scientists from diverse disciplines.
Beginning in 1987, core funding for the HBES has been provided through
the Long-term Ecological Research (LTER) network of the National Science
Foundation. Together with support from the USDA Forest Service, the LTER
program provides base funding for most monitoring activities in the
HBES. Individuals or groups of researchers are supported by competitive
grants to pursue a variety of specific research studies in the HBES in
cooperation with the LTER program. From 1963, over 1750 publications
have been produced through the HBES, providing a wealth of information
on the structure, function and development of forest, stream and lake
ecosystems.

For more history of Hubbard Brook Experimental Forest: Events Leading to
the Establishment of the HBEF, written by Jim Hornbeck, May 2001 (pdf)

\section{Conceptual Background}\label{conceptual-background}

The mission of the Hubbard Brook Ecosystem Study is to improve
understanding of the response of northern forest ecosystems to natural
and anthropogenic disturbances. In the overall conceptual model
underlying the Study (Figure~\ref{fig-conceptual-diagram}), three types
of disturbances - changing atmospheric chemistry, climate, and biota -
drive changes in the interacting components of the ecosystem, including
vegetation, biogeochemistry, hydrology, and food webs. These changes
play out on a template that includes the biogeophysical characteristics
of the landscape. The ecosystem responses may feedback to modify the
template and alter biotic change. The template and the functional
responses change through time in response to the disturbances and
internal drivers.

\begin{figure}

\centering{

\pandocbounded{\includegraphics[keepaspectratio]{figs/Intro/ConceptualDiagram.png}}

}

\caption{\label{fig-conceptual-diagram}Conceptual model underlying the
Hubbard Brook Ecosystem Study}

\end{figure}%

In Figure~\ref{fig-changing-chemistry} we illustrate measured and
hypothesized changes in ecosystem concentrations and fluxes in response
to long-term decreases in S and N atmospheric deposition and increases
in CO2 concentrations. Up and down arrows indicate increases and
decreases. Gold arrows reflect measured changes, purple arrows show
hypothesized changes. Gray arrows indicate major ecosystem processes
driving other responses in the ecosystem. We have observed increases in
soil pH and base saturation in response to declining S and N deposition
and expect these changes to increase soil P availability. We have also
observed increases in soil microbial biomass, decreases in forest floor
organic matter stocks, and shifts in solution N losses from inorganic to
organic forms. We hypothesize that rising atmospheric CO2 contributes to
these changes by stimulating plant CO2 uptake and nutrient demand for
growth, which in turn drives an increase in plant N and P uptake from
solution, increased nutrient resorption from litter and decreased litter
C/N ratios, as well as increases in allocation of plant C belowground to
spur organic matter mineralization and soil respiration.

\begin{figure}

\centering{

\pandocbounded{\includegraphics[keepaspectratio]{figs/Intro/Conceptual_ChangingChemistry.png}}

}

\caption{\label{fig-changing-chemistry}Illustration of measured and
hypothesized changes in ecosystem concentrations and fluxes in response
to long-term decreases in S and N atmospheric deposition and increases
in CO2 concentrations.}

\end{figure}%

Climate change disturbance has myriad effects on forest ecosystem
dynamics. Here we highlight (Figure~\ref{fig-climate-hydrology}) a
conceptual framework for diagnosing the cause of the recent observed
increase in rates of evapotranspiration (ET) at Hubbard Brook (see
Hydrology Chapter). We are investigating how atmospheric conditions and
vegetation interact nonlinearly to affect ET. Orange arrows indicate
observed trends in the long-term data. Question marks show where there
is uncertainty in our understanding. The green arrow shows how the
coupling of canopy and aerodynamic resistance affects rates of
evapotranspiration.

\begin{figure}

\centering{

\pandocbounded{\includegraphics[keepaspectratio]{figs/Intro/Conceptual_ClimateHydrology.png}}

}

\caption{\label{fig-climate-hydrology}A conceptual framework for
diagnosing the cause of the recent observed increase in rates of
evapotranspiration (ET) at Hubbard Brook.}

\end{figure}%

Our conceptual model of the hierarchical response of ecosystem function
to chronic stressors and episodic events is illustrated in
Figure~\ref{fig-changing-biota}). (A) Initial responses to stress
trigger species-specific responses with limited system-wide impact. (B)
With continued exposure, species abundances begin to re-order, resulting
in reduced ecosystem function. (C) This slow loss of function can be
accelerated by an (D) episodic disturbance. (E) If the stress persists,
further changes in the biota are expected. These disturbance-driven
shifts in biodiversity will have profound consequences on species
additions and losses, species demography and evolution, and energy flow
(inset).

\begin{figure}

\centering{

\pandocbounded{\includegraphics[keepaspectratio]{figs/Intro/Conceptual_ChangingBiota.png}}

}

\caption{\label{fig-changing-biota}The conceptual model of the
hierarchical response of ecosystem function to chronic stressors and
episodic events.}

\end{figure}%

Our conceptual model of the biogeophysical template on an idealized
hillslope (Figure~\ref{fig-biophysical-template}) shows five slope
positions along a depth-to-bedrock gradient that results in predictable
variation in soils, shallow groundwater movement, and saturation
frequency. Hydro-bio-pedo interactions along these soil-slope variations
structure vegetation and other biota, form linkages with temporary and
perennial streams, and shape the complex functional responses of
ecosystem components (Figure~\ref{fig-conceptual-diagram}).

\begin{figure}

\centering{

\pandocbounded{\includegraphics[keepaspectratio]{figs/Intro/Conceptual_BiophysicalTemplate.png}}

}

\caption{\label{fig-biophysical-template}The conceptual model of the
biogeophysical template on an idealized hillslope.}

\end{figure}%

\section{Research Approaches}\label{research-approaches}

The framework for the HBES is the small watershed approach which
provides precise quantitative water and element input-output budgets for
the forested ecosystems. In a catchment where the underlying bedrock is
impermeable, water falling on the catchment leaves only by
evapotranspiration (ET) or as stream discharge. By quantifying the
latter, annual ET can be calculated by difference:

Eq. 1. AET = Precipitation -- stream discharge

For chemical elements without a gas phase at Earth temperature, the
small watershed also allows estimation of soil weathering (Johnson et
al.~1981):

Eq. 2. Weathering = stream output -- atmospheric input + ∆ storage

Here, the stream output and atmospheric input of an element are
determined as the product of water flux and the element concentration in
the water. The ∆ storage term refers to any change in the stock or pool
of an element in the soil organic matter or vegetation biomass in the
ecosystem. For example, when vegetation biomass is aggrading some of the
chemicals weathered from soil minerals are accumulating in the trees
rather than lost in stream discharge and the ∆ storage term is positive
in Equation 2. Also, some elements enter the ecosystems as dry rather
than wet deposition (e.g.~dust particles), and this flux is included in
the atmospheric input term.

For elements with gaseous phases (e.g., N, C), separate measurements of
gas fluxes in and out of the ecosystem are needed to estimate material
budgets (see below). Experimental manipulations (e.g., forest harvest,
fertilization) can be applied at the small watershed scale to quantify
whole ecosystem responses (Bormann et al.~1974). The internal processes
determining the whole ecosystem responses are measured at the plot scale
at selected locations within or nearby the small watershed. Together
these measurements form the basis for much of the research in the HBES
including long-term monitoring as detailed below.

\section{Atmospheric Inputs}\label{atmospheric-inputs}

\begin{figure}

\centering{

\pandocbounded{\includegraphics[keepaspectratio]{figs/Intro/Rain_GaugeB.jpg}}

}

\caption{\label{fig-rain-gauge}Atmospheric deposition collectors at
Hubbard Brook.}

\end{figure}%

Atmospheric deposition of elements to the ecosystem is the sum of wet
and dry deposition. Measurement of wet deposition has been conducted on
a weekly basis since 1963 at a network of sampling sites across the HBEF
(Figure 7). All major dissolved solutes are measured and together with
precipitation water volume from standard rain gauges, element fluxes in
wet deposition are determined.

Dry deposition of particles and gases is more difficult to quantify than
wet deposition and can constitute a substantial proportion of the input
of several important elements, especially nitrogen and sulfur, that are
derived partly from pollutant emissions. Some of the dry deposition can
be captured on artificial samplers or as canopy throughfall, but the
calculation of dry deposition flux is complex and uncertain, and we
refer the reader to detailed explanations in the literature (Likens et
al.~2002).

\section{Stream Output}\label{stream-output}

\begin{figure}

\centering{

\pandocbounded{\includegraphics[keepaspectratio]{figs/Intro/Weir3a.jpg}}

}

\caption{\label{fig-weir-photo}Photograph of a V-notch weir at Hubbard
Brook}

\end{figure}%

Stream discharge of water from nine small watersheds in the HBEF is
measured continuously with 90o or 120o V-notch weirs located at the base
of the surveyed catchments (Figure~\ref{fig-weir-photo}). The weirs
continuously measure stage height in a stilling pond above the weir
outlet. Each weir is calibrated by directly measuring the relationship
between stage height and water flux. To calculate element fluxes, water
samples for chemical analysis are collected weekly just above each weir.
More frequent measurements to more precisely quantify storm-event fluxes
have been conducted periodically using automated samplers. Flux of
particulate matter is measured by removing accumulated material in the
weir stilling pond periodically. The flux of fine suspended solids
during large storm events also has been measured using permeable bags
positioned below the weir.

\section{Forest Vegetation}\label{forest-vegetation}

The composition, biomass and element content of the vegetation is
measured periodically within the gauged watersheds and throughout the
entire HBEF. The basic approach is to measure the diameter of trees
across the whole watershed (e.g., WS6, WS1) or in smaller plots usually
at five-year intervals. Forest biomass is estimated using allometric
equations relating DBH to other tree dimensions (e.g.~total biomass and
biomass of tissue components like leaves, branches, etc.). These
equations were developed at the site by directly measuring trees
harvested during experimental manipulations (Whittaker et al.~1974,
Siccama et al.~1994). The pool or stock of an element is estimated from
the product of mass and element concentration in various tissues. Thus,
changes in the stock of an element in the forest biomass are calculated
typically at a 5-year intervals.

\section{Soils}\label{soils-1}

The mass and element stocks of soils have been estimated using the
quantitative pit approach. The soil is carefully excavated manually from
a 0.5 m2 area through depth increments to the C horizon. Soil mass in
each depth increment is weighed and sieved in the field and subsamples
are taken for chemical analysis in the laboratory. Because of the large
stocks and high spatial variability of elements in soil, detection of
changes through time or in response to treatments is challenging.
Nevertheless, we have been able to estimate the net soil release for
major elements resulting from acid deposition and forest harvest on the
basis of quantitative pit sampling (Johnson et al.~1991).

\section{Gaseous Output}\label{gaseous-output}

The small watershed approach does not, by itself, account for outputs in
gas phase, an important flux pathways for the nutrient element nitrogen.
Gaseous flux of N remains a key uncertainty in the N budget at the HBEF
(Yanai et al.~2013). The principal gaseous flux of N results from
denitrification, a process that is notoriously variable in space and
time. The flux chamber approach, in which the accumulation of gaseous N
diffusing from soil is measured over short time intervals
(e.g.~minutes), is applied routinely to estimate flux of nitrous oxide
in several plots in and around the experimental watersheds at the HBEF.
Unfortunately, because of the high background in the atmosphere this
approach is not effective for measuring N2 flux. Laboratory, isotopic
and modeling approaches have provided indirect estimates of N2 flux from
soils (e.g., Wexler et al.~2014).

\begin{figure}

\centering{

\pandocbounded{\includegraphics[keepaspectratio]{figs/Intro/HB_FluxTower.jpg}}

}

\caption{\label{fig-flux-tower}Photograph of flux tower instrumentation
at Hubbard Brook.}

\end{figure}%

Fluxes of carbon dioxide and water vapor from the forest ecosystem can
be estimated using aerodynamic approaches with eddy flux towers.
Recently, an eddy flux tower has been installed in the lower valley at
the HBEF and began collecting data in 2016
(Figure~\ref{fig-flux-tower}). The facility will soon provide the first
direct estimates of gross primary production, autotrophic and total
ecosystem respiration, and net ecosystem productivity for the forest in
the footprint of the tower.

\section{Internal Element Fluxes}\label{internal-element-fluxes}

The processes regulating the biogeochemistry of the forest ecosystem
include a variety of major internal element fluxes which are monitored
in selected intensive plot locations in and around the experimental
watersheds at the HBEF. In particular, element leaching through soils is
measured using zero-tension lysimeters positioned beneath the forest
floor, Bh and Bs horizons. Water fluxes at each depth, estimated with a
hydrologic model, are used with measured element concentrations in soil
solutions to estimate soil leaching fluxes (Cho et al.~2010).

Litterfall is measured using permanent litter traps co-located in plots
with soil lysimeters. Litter is sorted by species, providing annual
estimates of leaf biomass and leaf area index in these plots.
Periodically, the chemistry of leaf litter is measured; combined with
chemistry of live foliage from the plots, these measurements provide
estimates of nutrient resorption from senescing foliage, a large
internal ecosystem flux especially for N and P (Ryan and Bormann 1982).
Deposition of woody litter as coarse woody debris is measured annually
on cleared 2.5 m2 plots co-located with litter traps. The chemistry of
canopy throughfall and element fluxes in canopy leaching also have been
measured in these intensive plots (Lovett et al.~1996) as well as the
biomass, chemistry and turnover of fine and coarse roots, using coring
and rhizotron methods, so that a comprehensive accounting of the
principal internal element fluxes is available for the forest ecosystems
in the experimental watersheds.

\section{References}\label{references}

Bormann, F. H., Likens, G. E., Siccama, T. G., Pierce, R. S., \& Eaton,
J. S. (1974). The export of nutrients and recovery of stable conditions
following deforestation at Hubbard Brook. Ecological Monographs, 44(3),
255--277. https://doi.org/10.2307/1942311

Cho, Y., Driscoll, C. T., Johnson, C. E., \& Siccama, T. G. (2010).
Chemical changes in soil and soil solution after calcium silicate
addition to a northern hardwood forest. Biogeochemistry, 100, 3--20.
https://doi.org/10.1007/s10533-009-9397-6

Fahey, T. J., Templer, P. H., Anderson, B. T., Battles, J. J., Campbell,
J. L., Driscoll, C. T., Fusco, A. R., Green, M. B., Kassam, K. A. S.,
Rodenhouse, N. L., \& Rustad, L. (2015). The promise and peril of
intensive site-based ecological research: Insights from the Hubbard
Brook ecosystem study. Ecology, 96(4), 885--901.
https://doi.org/10.1890/14-1043.1

Johnson, C. E., Johnson, A. H., Huntington, T. G., \& Siccama, T. G.
(1991). Whole-tree clear-cutting effects on soil horizons and
organic-matter pools. Soil Science Society of America Journal, 55,
497--502. https://doi.org/10.2136/sssaj1991.03615995005500020042x

Johnson, N. M., Driscoll, C. T., Eaton, J. S., Likens, G. E., \&
McDowell, W. H. (1981). Acid rain, dissolved aluminum and chemical
weathering at the Hubbard Brook Experimental Forest, New Hampshire.
Geochimica et Cosmochimica Acta, 45(9), 1421--1437.
https://doi.org/10.1016/0016-7037(81)90276-3

Likens, G. E., Driscoll, C. T., Buso, D. C., Mitchell, M. J., Lovett, G.
M., Bailey, S. W., Siccama, T. G., Reiners, W. A., \& Alewell, C.
(2002). The biogeochemistry of sulfur at Hubbard Brook. Biogeochemistry,
60(3), 235--316. https://doi.org/10.1023/A:1020972100496

Lovett, G. M., Nolan, S. S., Driscoll, C. T., \& Fahey, T. J. (1996).
Factors regulating throughfall flux in a New Hampshire forested
landscape. Canadian Journal of Forest Research, 26, 2134--2144.
https://doi.org/10.1139/x26-241

Ryan, D. F., \& Bormann, F. H. (1982). Nutrient resorption in northern
hardwood forests. BioScience, 32(1), 29--32.
https://doi.org/10.2307/1308541

Siccama, T. G., Hamburg, S. P., Arthur, M. A., Yanai, R. D., Bormann, F.
H., \& Likens, G. E. (1994). Corrections to allometric equations and
plant tissue chemistry for Hubbard Brook Experimental Forest. Ecology,
75(1), 246--248. https://doi.org/10.2307/1939407

Wexler, S. K., Goodale, C. L., McGuire, K. J., Bailey, S. W., \&
Groffman, P. M. (2014). Isotopic signals of summer denitrification in a
northern hardwood forested catchment. Proceedings of the National
Academy of Sciences, 111(46), 16413--16418.
https://doi.org/10.1073/pnas.1404321111

Whittaker, R. H., Bormann, F. H., Likens, G. E., \& Siccama, T. G.
(1974). The Hubbard Brook ecosystem study: Forest biomass and
production. Ecological Monographs, 44(2), 233--254.
https://doi.org/10.2307/1942313

Yanai, R. D., Vadeboncoeur, M. A., Hamburg, S. P., Arthur, M. A., Fuss,
C. B., Groffman, P. M., Siccama, T. G., \& Driscoll, C. T. (2013). From
missing source to missing sink: Long-term changes in the nitrogen budget
of a northern hardwood forest. Environmental Science \& Technology,
47(20), 11440--11448. https://doi.org/10.1021/es402572j




\end{document}
